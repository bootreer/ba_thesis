\chapter{Evaluation}\label{chapter:eval}
TODO: plots!!!

\section{Setup}
All benchmarks are run on a system with an Intel Xeon E5-2660 with 251GiB of RAM running Ubuntu 23.10 with a 1TB Samsung Evo 970 Plus NVMe SSD.

\begin{table}
    \centering
    \begin{tabular} { ||c|c|c|| }
        \cline{1-3}
        Sequential read & \multicolumn{2}{|c||}{3500 MB/s} \\ \cline{1-3}
        Sequential write & \multicolumn{2}{|c||}{3300 MB/s} \\ \cline{1-3}
        \multirow{2}{*}{Queue Depth 1, Thread 1} & Random read & 19000 IOPS \\ \cline{2-3}
                                                    & Random write & 60000 IOPS \\ \cline{1-3}
        \multirow{2}{*}{Queue Depth 32, Thread 4} & Random read & 600K IOPS \\ \cline{2-3}
                                                    & Random write & 550K IOPS \\ \cline{1-3}
    \end{tabular}
    \caption{Samsung Evo 970 Plus performance limits as written in the datasheet}
    \label{tab:evoplus}
\end{table}

In the following sections we will compare our driver's performance with the datasheet numbers in \autoref{tab:evoplus}, as well as against other I/O engines: \texttt{libaio}, \texttt{io\_uring}, \texttt{SPDK} and the Linux file I/O API \texttt{pread}/\texttt{pwrite} (\texttt{psync}). Each I/O engine is tested by running a read or write workload over 900 seconds, with I/O unit sizes of 4KiB. All write benchmarks are done on an empty drive.

\section{Throughput}

\section{Latency}

\section{Comparison with other I/O engines}
\subsection{Queue depth 1}


\subsection{Queue depth 32, Multi-threaded}
