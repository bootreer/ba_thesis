\chapter{Conclusion}\label{c:c}
In this thesis, we have presented vroom, a novel user space and poll-based NVMe driver written in Rust. Our evaluations have shown here that developing a user space NVMe driver in a higher-level programming language with SPDK-like performance is feasible while simultaneously containing fewer lines of driver code overall. We reach comparable throughput speeds as SPDK while outperforming the Linux kernel I/O APIs by avoiding the kernel altogether.

\paragraph{Future Work}
There are still many aspects in the driver where optimisations can be investigated, such as the effect of prefetching pages for I/O operations or implementing IOMMU support, so root privileges are not required to start the driver and measure the performance impact it may have. Investigating the performance impacts of interrupts compared to polling and comparing different interrupt methods are also open topics.

As vroom does not support all NVMe features, we can still add many features to the driver, like supporting SGLs and comparing them to PRP lists or using the NVMe controller's onboard memory buffer.

With \texttt{libaio} not being actively worked on, the last commit being made two years ago \cite{libaio-source}, and \texttt{io\_uring} exhibiting enough security concerns that Google disabled the use of the storage engine on all production servers \cite{google-iou}, and the containerd runtime disabling \texttt{io\_uring} system calls \cite{containerdeez-nuts} altogether, there lacks an I/O engine which is safe, performant and simple to use at the same time; extending vroom by a block device and filesystem layer could serve to fill this gap.
