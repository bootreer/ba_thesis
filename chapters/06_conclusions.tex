\chapter{Conclusion}\label{c:c}
In this thesis we have presented vroom, a novel user space, poll-based NVMe driver written in Rust. What our evaluations have shown here is that it is very much feasible to develop a user space NVMe driver in a higher level programming language with SPDK-like performance, while at the same time containing fewer lines of driver code overall. We reach comparable throughput speeds as SPDK, while outperforming the Linux kernel I/O APIs by avoiding the kernel altogether.

\paragraph{Future Work}
There are still many aspects in the driver where optimisations can be investigated, such as the effect of prefetching pages for I/O operations, or implementing IOMMU support, so root priviledges are not required to start the driver, and measuring the performance impact it may have. Investigating the performance impacts interrupts have compared to polling, as well as comparing different interrupt methods are open topics as well.

As vroom does not support all NVMe features, there are still many features we can add to the driver, like supporting SGLs and comparing it to PRP lists, or using the NVMe controller's onboard memory buffer.

With \texttt{libaio} not being actively worked on, the last commit being made two years ago \cite{libaio-source}, and \texttt{io\_uring} exhibiting enough security concerns that Google disabled the use of the storage engine on all production servers \cite{google-iou}, while the containerd runtime has disabled \texttt{io\_uring} system calls \cite{containerdeez-nuts}, there lacks an I/O engine which is safe, performant and simple to use at the same time; extending vroom by a block device and filesystem layer could serve to fill this gap.
