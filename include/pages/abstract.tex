\chapter{Abstract}
The exponential growth in the performance of Non-Volatile Memory Express (NVMe) Solid State Drives (SSDs) has led to a need for advanced software mechanisms capable of fully leveraging their potential. Traditional Linux I/O APIs and even newer asynchronous APIs, while capable, often fall short in achieving the lowest possible latency and highest throughput due to their dependence on kernel-based I/O paths which introduce significant overheads. The Storage Performance Development Kit (SPDK) offers a solution through its user-space driver model, eliminating this overhead, but at the cost of increased complexity and potential safety concerns due to its C codebase.

Recognising these challenges, this thesis proposes the development of an NVMe driver written in Rust, a language that promises memory safety without sacrificing performance, aiming to harness the full capabilities of NVMe SSDs with a simpler and safer approach. The work undertakes a comparative analysis between the proposed Rust driver, SPDK, as well as the Linux I/O APIs, with the goal of simplifying access to high-performance storage technologies.

% lol weiß nicht ganz
% Ultimately, this thesis contributes to the ongoing discourse on the optimization of storage I/O pathways, addressing the dual challenges of maximizing performance while minimizing complexity and safety risks. Through its exploration of a Rust-based approach to NVMe driver development, we seek to pave a way for future research and development in safe, efficient, and user-friendly storage systems.
