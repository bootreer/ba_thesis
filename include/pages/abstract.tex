\chapter{Abstract}
Today's SSDs are capable of performing millions of IOPS. However, while capable, traditional Linux I/O APIs and even newer asynchronous APIs often fall short in achieving the lowest possible latency and highest throughput due to their dependence on kernel-based I/O paths which introduce significant overheads. The Storage Performance Development Kit (SPDK) offers a solution through its user-space driver model, eliminating this overhead, but at the cost of increased complexity and potential safety concerns due to its C codebase.

Recognising these challenges, we propose the development of an NVMe driver written in Rust, a language that promises memory safety without sacrificing performance, aiming to harness the full capabilities of NVMe SSDs in a simpler and safer approach. We present a novel user space driver employing zero-copy I/O and simple abstractions, enabling an easier way to assess individual NVMe features and I/O path optimisations. We show that, despite the stripped-down design of the driver, we achieve SPDK-like throughput and latency. Our work undertakes a comparative analysis between vroom, our proposed NVMe driver, and SPDK, as well as the Linux I/O APIs, with the goal of simplifying access to high-performance storage technologies.
